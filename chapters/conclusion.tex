\chapter{Sklepne ugotovitve}
% \chapter{Conclusion} % Swap if writing in English

Uporaba \LaTeX{a} je v okviru Diplomskega seminarja \textbf{obvezna}!
Izbira -- \LaTeX\ ali ne \LaTeX\ -- pri pisanju dejanske diplomske naloge pa je pre\-pu\-šče\-na
dogovoru med diplomantom in njegovim mentorjem.

Res je, da so prvi koraki v \LaTeX{}u težavni.
Ta dokument naj služi kot začetna opora pri hoji.
Pri kakršnihkoli nadaljnih vprašanjih ali napakah pa svetujemo uporabo Googla, saj je spletnih
strani za pomoč pri odpravljanju težav pri uporabi \LaTeX{}a ogromno.

Preden diplomo oddate na sistemu STUDIS, še enkrat preverite, če so slovenske besede, ki vsebujejo
črke s strešicami, pravilno deljene in da ne segajo preko desnega roba.
Poravnavo po vrsticah lahko kontrolirate tako, da izvorno datoteko enkrat testno prevedete z opcijo
\texttt{draft}, kar vam pokaže predolge vrstice.
