\chapter{Koristni nasveti pri pisanju v \LaTeX{u}}
\label{latex}

Programski paket \LaTeX\ je bil prvotno predstavljen v priročniku~\cite{lamport} in je v resnici
nadgradnja sistema \TeX\ avtorja Donalda Knutha~\cite{knuth}, znanega po svojih knjigah o
umetnosti programiranja ter Knuth-Bendixovem algoritmu~\cite{knuth1983simple}.
\TeX\ in njegove izpeljanke so odprtokodni programi.

Različnih implementacij \LaTeX{}a je cela vrsta.
Za OS X priporočamo TeXShop, za Windows PC pa MikTeX.
Spletna verzija, ki poenostavi sodelovanje pri pisanju, je Overleaf.

Včasih smo si pri pisanju v \LaTeX{}u pomagali predvsem s tiskanimi pri\-ro\-čni\-ki~\cite{lamport},
danes pa je enostavneje in hitreje, da ob vsakem problemu za pomoč enostavno povprašamo Google,
saj je na spletu cela vrsta forumov za pomoč pri \TeX iranju.

\LaTeX\ včasih ne zna pravilno deliti slovenskih besed, ki vsebujejo črke s streši\-ca\-mi.
Če taka beseda štrli preko desnega roba, lahko \LaTeX{}u pokažemo, kje se tako besedo deli takole:
\verb=ra\-ču\-nal\-ni\-štvo=.
Katere vrstice so predolge lahko vidimo tako, da dokument prevedemo s vključeno opcijo
\texttt{draft}: \verb=\documentclass[a4paper, 12pt, draft]{book}=.

Predlagamo, da v izvornem besedilu začenjate vsak stavek v novi vrstici, saj \LaTeX\ sam
razporeja besede po vrsticah postavljenega besedila.
Bo pa zato iskanje po izvornem besedilu in popravljanje veliko hitrejše.
Večina sistemov za \TeX{}iranje sicer omogoča s klikanjem enostavno prestopanje iz prevedenega
besedila na ustrezno mesto v izvornem besedilu in obratno.

Boljšo preglednost dosežemo tako kot pri pisanju programske kode -- z vizualnim urejanjem kode
in izpuščanjem praznih vrstic.
Pri spreminjanju in dodajanju izvornega besedila je najbolje pogosto prevajati, da se sproti
prepričamo, če so naši nameni pravilno izpolnjeni.

Kadar besedilo, ki je že bilo napisano z nekim vizualnim urejevalnikom (npr. z Wordom), želimo
prenesti v \LaTeX, je tudi najbolje to delati postopoma s posameznimi bloki besedila, tako da
lahko morebitne napake hitro identificiramo in odpravimo.
Za prevajanje Wordovih datotek v \LaTeX\ -- in obratno -- sicer obstajajo prevajalniki, ki pa
običajno ne generirajo tako čisto logično strukturo besedila, kot jo sicer \LaTeX\ omogoča.
Hiter in enostaven način prevedbe besedila, ki  zahteva sicer ročne dopolnitve, lahko poteka
tudi tako, da besedilo urejeno z vizualnim urejevalnikom najprej shranimo v formatu pdf,
nato pa to besedilo uvozimo v urejevalnik, kjer urejamo izvorno besedilo v formatu \LaTeX.

\section{Pisave v \LaTeX u}

V  \LaTeX ovem okolju lahko načeloma uporabljamo poljubne pisave.
Izbira poljubne pisave pa ni tako enostavna kot v vizualnih urejevalnikih besedil.
Posamezne oblikovno medseboj usklajene pisave so običajno združene v družine pisav.
V \LaTeX u se privzeta družina pisav imenuje Computer Modern, kjer so poleg navadnih črk
(roman v \LaTeX u) na voljo tudi kurzivne črke (\textit{italic} v \LaTeX u),
krepke (\textbf{bold} v \LaTeX u), kapitelke (\textsc{small caps} v \LaTeX u), linearne črke
({\textsf{san serif} v \LaTeX u}), pisava pisalnega stroja (\texttt{typewriter} v \LaTeX u) in
nekatere njihove kombinacije, npr. krepke linearne črke ({\textbf{\textsf{san serif}} v \LaTeX u}).
V istem dokumentu zaradi skladnega izleda uporabljamo običajno le pisave ene družine.
Pomembna je tudi konsistentna raba večih pisav in da ne pretiravamo z mešanjem različnih pisav.

Ko začenjamo uporabljati \LaTeX, je zato najbolj smiselno uporabljati kar privzete pisave,
s katerimi je napisan tudi ta dokument.
Z ustreznimi ukazi  lahko nato preklapljamo med navadnimi, kurzivnimi, krepkimi in drugimi pisavami.
Zelo enostavna je tudi izbira velikosti črk.
\LaTeX\ odlično podpira večjezičnost, tudi v sklopu istega dokumenta, saj obstajajo pisave za
praktično vse jezike, tudi take, ki ne uporabljajo latinskih črk.

Za prikaz programske kode se pogosto uporablja pisava, kjer imajo vse črke enako širino, kot so
črke na mehanskem pisalnem stroju ({\texttt{typewriter} v \LaTeX u}).

Najbolj priročno okolje za pisanje kratkih izsekov programske kode je okolje \texttt{verbatim},
saj ta ohranja vizualno organizacijo izvornega besedila in ima privzeto pisavo pisalnega stroja.

\begin{verbatim}
for (i = 0; i < 100; i++)
   for (j = i; j < 10; j++)
      some_function(i, j);
\end{verbatim}
