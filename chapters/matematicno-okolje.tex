\chapter{Matematično okolje in sklicevanje na besedilne konstrukte}

Matematična ali popolna indukcija je eno prvih orodij, ki jih spoznamo za dokazovanje trditev pri
matematičnih predmetih.
\begin{izrek}
    \label{iz:1}
    Za vsako naravno število $n$ velja
    \begin{equation}
        n < 2^n.
        \label{eq:1}
    \end{equation}
\end{izrek}
\begin{dokaz}
    Dokazovanje z indukcijo zahteva, da neenakost~\eqref{eq:1} najprej preverimo za najmanjše
    naravno število -- $0$.
    Res, ker je $0 < 1 = 2^0$, je neenačba~\eqref{eq:1} za $n=0$ izpolnjena.

    Sledi indukcijski korak. S predpostavko, da je neenakost~\eqref{eq:1} veljavna pri nekem
    naravnem številu $n$, je potrebno pokazati, da je ista neenakost v veljavi tudi pri njegovem
    nasledniku -- naravnem številu $n+1$.
    Računajmo.
    \begin{align}
        n+1 & < 2^n + 1       \label{eq:2} \\
            & \le 2^n + 2^n \label{eq:3}   \\
            & = 2^{n+1}       \nonumber
    \end{align}
    Neenakost~\eqref{eq:2} je posledica indukcijske predpostavke, neenakost~\eqref{eq:3} pa
    enostavno dejstvo, da je za vsako naravno število $n$ izraz $2^n$ vsaj tako velik kot 1.
    S tem je dokaz Izreka~\ref{iz:1} zaključen.
\end{dokaz}

Opazimo, da je \LaTeX\ številko izreka podredil številki poglavja.
Na podoben način se lahko s pomočjo ukazov \verb|\label| in \verb|\ref| sklicujemo tudi na druge
besedilne konstrukte, kot so med drugim poglavja, podpoglavja in plovke, ki jih bomo spoznali v
naslednjem poglavju.
