\chapter{Kaj pa literatura?}
\label{lit}

Kot smo omenili že v uvodu, je pravi način za citiranje literature uporaba BibLaTeXa~\cite{biblatex}.
BibLaTeX\ zagotovi, da pri določeni vrsti literature ne izpustimo nobene obvezne informacije
in da vse informacije dosledno navajamo na enak način in po istem vrstnem redu.
BibLaTeX\ je nadgradnja starejšega sistema BibTeX.
Novejši sistem je bolje prilagojen slovenščini in navajanju spletnih virov.
Sicer pa so starejše datoteke \texttt{.bib} kompatibilne z BibLaTeX om.

Osnovna ideja BibLaTeXa je, da vse informacije o literaturi zapisujemo v posebno datoteko,
v našem primeru je to \texttt{literatura.bib}.
Vsakemu viru v tej datoteki določimo simbolično ime.
V našem primeru je v tej datoteki nekaj najbolj značilnih zvrsti literature, kot so knjige~\cite{lamport},
članki v revijah~\cite{leonardo} in zbornikih konferenc~\cite{ciuha2010visualization}, poglavja v
knjigah~\cite{poglavje_springer}, spletni viri~\cite{slovarji,video}, tehnično
poročilo~\cite{andersen2012kinect}, diplome~\cite{diploma} itd.
Diploma~\cite{diploma} iz leta 1990 je bila prva diploma na tedanji Fakulteti za elektrotehniko
in računalništvo, ki je bila oblikovana z \LaTeX om!
Reference, ki so na spletnih straneh arhivirane v elektronski obliki, imajo običajno \v stevilko
DOI (\url{http://dx.doi.org}), ki jo zato tudi vključimo v izpis literature in tako bralcu
elektronske verzije naše publikacije ponudimo neposredno povezavo do elektronske kopije te reference.

Po vsaki spremembi pri sklicu na literaturo moramo najprej prevesti izvorno besedilo s prevajalnikom
\LaTeX, nato s prevajalnikom BibLaTeX, ki ustvari datoteko {\tt vzorec\_dip\_Seminar.bbl},
in nato še dvakrat s prevajalnikom \LaTeX.
V okolju Overleaf je to večkratno prevajanje z različnimi prevajalniki uporabniku skrito.
Zato tudi začetnim uporabnikom \LaTeX a svetujemo uporabo Overleafa.

Kako se spisek literature nato izpiše (ali so posamezni viri razvrščeni po vrstnem redu sklicevanja,
ali po abecedi priimkov prvih avtorjev, ali se imena avtorjev pišejo pred priimki itd.) je odvisno
od parametrov paketa BibLaTeX.
V diplomi bomo uporabili parametre \texttt{style=numeric}, kar pomeni, da bodo sklici na literaturo
v besedilu označeni z zaporednimi številkami, za vrstni red izpisa referenc pa \texttt{sorting=nty},
kar pomeni, da bodo reference urejene po priimkih prvih avtorjev, nato po naslovu reference in
nazadnje po letu izdaje~\cite{ctan}.
Zato je potrebno pri določenih zvrsteh literature, ki nima avtorjev, dodati parameter \texttt{key},
ki določi vrstni red vira po abecedi.

Ko začenjamo uporabljati BibLaTeX\ je lažje, če za urejanje datoteke \texttt{.bib} uporabljamo kar
isti urejevalnik kot za urejanje datotek \texttt{.tex}, čeprav obstajajo tudi posebni urejevalniki
oziroma programi za delo z datotekami \texttt{.bib}.

Le če se  na določen vir v besedilu tudi sklicujemo, se bo ta vir pojavil tudi v spisku literature.
Tako je avtomatično zagotovljeno, da se na vsak vir v seznamu literature tudi sklicujemo v besedilu diplome.
V datoteki \texttt{.bib} imamo sicer lahko veliko več virov za literaturo, kot jih bomo uporabili v diplomi.


\section{Zbiranje virov za seznam literature}

Vire v formatu \texttt{.bib} lahko enostavno poiščemo in prekopiramo iz spletnih strani založnikov
ali različnih akademskih spletnih portalov za iskanje znanstvene literature.
Izvoz referenc v Google učenjaku še dodatno poenostavimo, če v nastavitvah izberemo BibTeX\ kot
želeni format za izvoz navedb.
Navedbe, ki jih prekopiramo iz Google učenjaka in drugih podobnih akademskih portalov, moramo pred
uporabo nujno preveriti, saj so taki navedki pogosto generirani povsem avtomatično in lahko
vsebujejo napačne ali nepopolne podatke.
Najpogosteje je napačen tip publikacije!

Pri sklicevanju na literaturo na koncu stavka moramo paziti, da je pika po ukazu \verb=\cite{ }=.
Da \LaTeX\ ne bi delil vrstico ravno tako, da bi sklic na literaturo v oglatih oklepajih začel novo
vrstico, lahko pred sklicem na literaturo dodamo nedeljiv presledek: \verb=~\cite{ }=.

Običajno se v besedilu sklicujemo  na nek vir ali več virov na koncu trdilnega stavka.
Kadar pa omenimo avtorja nekega vira, pa sklic običajno vstavimo za njegovim priimkom.

Dandanes se skoraj vsi pri iskanju informacij vedno najprej lotimo iskanja preko svetovnega spleta.
Rezultati takega iskanja pa so pogosto spletne strani, ki danes obstajajo, jutri pa jih morda ne bo
več, ali pa vsaj ne v taki obliki, kot smo jo prebrali.
Smisel navajanja literature pa je, da tudi po dolgih letih nekdo, ki bo bral vašo diplomo, lahko
poišče vire, ki jih navajate v diplomi.

Znanstveni rezultati, ki so objavljeni v obliki recenziranih člankov, bodisi v konferenčnih
zbornikih, še bolje pa v znanstvenih revijah, so veliko bolj izčiščen in zanesljiv vir informacij,
saj so taki članki šli skozi recenzijske postopke.
Predvsem pa so taki članki stabilen vir informacij, saj se načeloma po njihovi objavi ne spreminjajo več.
Skoraj vsi ti članki so dandanes dosegljivi tudi v elektronski obliki, bodisi v arhivih založnikov,
univerzitetnih repozitorijih ali tudi na osebnih spletnih straneh njihovih avtorjev.
Zato na svetovnem spletu začenjamo iskati vire za strokovna besedila predvsem preko akademskih
spletnih portalov, kot so npr. Google učenjak, Research Gate ali Academia, saj so na teh portalih
rezultati iskanja le akademske publikacije.
Če je za dostop do nekega članka potrebno plačati, se obrnemo za pomoč in dodatne informacije na našo knjižnico.

Za označevanje člankov, ki so na voljo v elektronski obliki, se je v zadnjem času uveljavila oznaka DOI
(\url{https://www.doi.org}), kar močno olajša iskanje teh referenc na spletu.
Založniki tudi starejšim člankom, ki so na voljo v elektronski obliki, za nazaj določajo oznake DOI.
Zato poskusite poiskati ustrezno oznako DOI za vsak članek, ki ga citirate in jo vključite v seznam literature.

Če res ne gre drugače, pa je pomembno, da pri sklicevanju na običajni spletni vir vedno navedemo
tudi datum, kdaj smo dostopali do tega vira.

Z uporabo BibLaTeXa je možno natisniti seznam literature posebej za določene vrste referenc,
na primer za članke v znanstvenih revijah, članke v konferenčnih zbornikih in poglavja v knjigah,
kot je prikazano tudi v tem vzorcu diplome.
Pri zbiranju literature si zato prizadevajte čimbolj napolniti sezname teh treh vrst referenc.

