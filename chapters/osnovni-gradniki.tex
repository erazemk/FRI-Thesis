\chapter{Osnovni gradniki \LaTeX{a}}

\LaTeX\ bi lahko najbolj preprosto opisali kot programski jezik namenjen oblikovanju besedil.
Tako kot vsak visokonivojski programski jezik ima tudi \LaTeX\ številne ukaze za oblikovanje
besedila in okolja, ki omogočajo strukturiranje besedila.

Vsi \LaTeX ovi ukazi se začnejo z levo poševnico \verb=\=, okolja pa definiramo bodisi s parom
zavitih oklepajev \{ in \} ali z ukazoma \verb=\begin{ }= in \verb=\end{ }=.
Ukazi imajo lahko tudi argumente, obvezni argumenti so podani v zavitih oklepajih, opcijski
argumenti pa v oglatih oklepajih.

Z ukazi torej definiramo naslov in imena avtorjev besedila, poglavja in podpoglavja in po
potrebi bolj podrobno strukturiramo besedila na spiske, navedke itd.
Posebna okolja so namenjena zapisu matematičnih izrazov, kratki primeri so v naslednjem poglavju.

Vse besedilne konstrukte lahko poimenujemo in se s pomočjo teh imen nato kjerkoli v besedilu
na njih tudi sklicujemo.

\LaTeX\ sam razporeja besede v odstavke tako, da optimizira razmike med besedami v celotnem odstavku.
Nov odstavek začnemo tako, da izpustimo v izvornem besedilu prazno vrstico. Da besedilo skoči
v novo vrstico pa ukažemo z dvema levima poševnicama.
Število presledkov med besedami v izvornem besedilo ni pomembno.
