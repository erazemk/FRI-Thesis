\chapter{Skladnost s standardom PDF/A}
\label{PDF}

Elektronsko verzijo diplome je potrebno oddati preko sistema STUDIS v formatu PDF/A
~\cite{howtopdfa,pdfa}, natančneje v formatu PDF/A-1b.
PDF/A format je namenjen dolgoročnemu arhiviranju elektronskih dokumentov.
Dokument v formatu PDF/A mora vsebovati vse potrebne informacije za prikazovanje in tiskanje dokumenta.
To pomeni, da mora dokument vsebovati vso besedilo, vse slike, fonte in barvne informacije.
Prva verzija standarda PDF/A, to je PDF/A-1 je bil objavljena leta 2005.
Standard PDF/A-1 določa dva nivoja skladnosti: PDF/A-1a in PDF/A-1b.
Nivo a (accesible) mora ustrezati vsem zahtevam standarda.
Nivo b (basic) pa zahteva le, da se ohrani vizualni izgled dokumenta.
Diplome, ki jih je potrebno oddati na sistemu STUDIS, morajo ustrezati nivoju standarda PDF/A-1b.

\LaTeX\ in omenjeni format imata še nekaj težav s sobivanjem.
Paket \texttt{pdfx.sty}, ki naj bi \LaTeX{u} omogočal podporo formatu PDF/A ne deluje vedno
v skladu s pričakovanji.

Zato raje priporočamo uporabo enega od mnogih spletnih mest, ki omo\-go\-ča\-jo konverzijo pdf
datotek v obliko, ki je skladna s standardom PDF/A-1b, npr. \url{https://pdf.online/pdf-to-pdfa},
kjer je možno tudi testirati, ali je neka pdf datoteka skladna s tem standardom.

V predlogi so poleg izvornega dokumenta \texttt{diploma-FRI-vzorec.tex}, še vložena slika
\texttt{galaksija.jpeg}, datoteka \texttt{literatura.bib} za uporabljeno literaturo ter
ikone za licenco Creative Commons.
