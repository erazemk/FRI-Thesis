\chapter{Pogoste napake pri pisanju v slovenščini}
\label{slo}

V slovenščini moramo paziti  pri uporabi pridevnikov, ki se ne sklanjajo, kot so npr. kratice.
Pravilno pišemo ``model CAD'' in \textbf{ne} ``CAD model''!

Pri sklanjanju tujih imen ne uporabljamo vezajev, pravilno je Applov operacijski sistem in
\textbf{ne} Apple-ov.

Pika, klicaj in vprašaj so levostični: pred njimi ni presledka, za njimi pa je presledek.
Klicajev in vprašajev se v strokovnih besedilih načeloma izogibamo.
Oklepaji so desnostični in zaklepaji levostični: (takole).

Z narekovaji označujemo premi govor, naslove, citate ali pa z njim dajemo besedam poseben pomen.
Narekovaji so stična ločila.
Ločimo različne narekovaje, vendar je v \LaTeX u najbolj enostavno uporabiti ``dvojni narekovaj zgoraj''.
Za druge vrste narekovajev je potrebno uvoziti dodatne pakete ali fonte.
Besede lahko vizualno označimo tudi z uporabo drugih pisav  iz iste družine, npr. kurzivno in
krepko pisavo, vendar pri uporabi teh fontov ne smemo pretiravati.

Vezaj je levo in desno stičen, npr. \verb=slovensko-angleški slovar= in ga pišemo z enim znakom za pomišljaj.
V slovenščini je presledek pred in po pomišljaju: Pozor -- hud pes! (\verb=Pozor -- hud pes!=).
V angleščini pa je za razliko pomišljaj levo in desno stičen in se v \LaTeX u piše s tremi
pomišljaji: \verb=---=.
S stičnim pomišljajem pa lahko nadomeščamo predlog od \dots do, denimo pri navajanju strani,
npr. preberite strani 7--11 (\verb=7--11=).

``Pred ki, ko, ker, da, če vejica skače''.
To osnovnošolsko pravilo smo v življenju po potrebi uporabljali, dopolnili, morda celo pozabili.
Pravilo sicer drži, ampak samo če je izpolnjenih kar nekaj pogojev (npr. da so ti vezniki
samostojni, enobesedni, ne gre za vrivek itd.).
Povedki so med seboj ločeni z vejicami, razen če so zvezani z in, pa, ter, ne–ne, niti–niti,
ali, bodisi, oziroma.
Sicer pa je bolje pisati kratke stavke kot pretirano dolge.

V računalništvu se stalno pojavljajo novi pojmi in nove besede, za katere pogosto še ne obstajajo
uveljavljeni slovenski izrazi.
Kadar smo v dvomih, kateri slovenski izraz je primeren, si lahko pomagamo z iskanjem na kakšnem
od slovenskih spletnih slovarjev~\cite{slovarji}, še posebej v \textit{Islovarju} Slovenskega
društva Informatika \cite{Islovar} in Slovarju Slovenskega društva za razpoznavanje vzorcev~\cite{sdrv}.
Sicer pa glavni vir za reševanje slovenskih jezikovnih zadreg spletišče \textit{Fran}~\cite{fran}.
