\chapter{Plovke: slike in tabele}
\label{ch2}

Slike in daljše tabele praviloma vključujemo v dokument kot plovke.
Pozicija plovke v končnem izdelku ni pogojena s tekom besedila, temveč z izgledom strani.
\LaTeX\ bo skušal plovko postaviti samostojno, praviloma na mestu, kjer se pojavi v izvornem
besedilu, sicer pa na vrhu strani, na kateri se na takšno plovko prvič sklicujemo.
Pri tem pa bo na vsako stran končnega izdelka želel postaviti tudi sorazmerno velik del besedila.
V skrajnem primeru, če imamo res preveč plovk na enem mestu besedila, ali če je plovka previsoka,
se bo \LaTeX\ odločil za stran popolnoma zapolnjeno s plovkami.

Poleg tega, da na položaj plovke vplivamo s tem, kam jo umestimo v izvorno besedilo, lahko na
položaj plovke na posamezni strani prevedenega besedila dodatno vplivamo z opcijami
\texttt{here, top} in \texttt{bottom}.
Zelo velike slike je najbolje postaviti na posebno stran z opcijo \texttt{page}.
Skaliranje slik po njihovi širini lahko prilagodimo širini strani tako, da kot enoto za širino
uporabimo kar širino strani, npr. \verb=0.5\textwidth= bo raztegnilo sliko na polovico širine strani.
Sliko lahko po potrebi tudi zavrtimo za 90 stopinj in jo razstegnemo na višino strani.
Tako bodo podrobnosti na sliki lažje berljive in prostor na strani bo bolje izkoriščen.

Na vse plovke se moramo v besedilu sklicevati, saj kot beseda plovka pove, plovke plovejo po
besedilu in se ne pojavijo točno tam, kjer nastopajo v izvornem besedilu.
Vendar naj bosta  sklic na plovko v besedilu in sama plovka v oblikovanem besedilu čim bližje
skupaj, tako da bralcu ne bo potrebno listati po diplomi.
Upoštevajte pa, da se naloge tiska dvostransko in da se hkrati vidi dve strani v dokumentu!
Na to, kje se bo slika ali druga plovka pojavila v postavljenem besedilu lahko torej najbolj
vplivamo tako, da v izvorni kodi plovko premikamo po besedilu nazaj ali naprej!

Tabele je najbolje oblikovati kar neposredno v \LaTeX u, saj za oblikovanje tabel obstaja zelo
fleksibilno okolje \texttt{tabular}.
Slike pa je po drugi strani  pogosto najla\v zje oblikovati oziroma izdelati z drugimi orodji
in programi, rezultate shraniti v formatu {\tt .pdf} ali {\tt .jpeg} in nato v \LaTeX u le
vključiti ustrezno slikovno datoteko. Za pisanje besed, ki so vključene v slike, uporabite
pisave/fonte, ki so čimbolj podobne pisavam v samem besedilu.

Knjižnica \url{https://en.wikibooks.org/wiki/LaTeX/PGF/TikZ}
pa omo\-go\-ča risanje raznovrstnih grafov neposredno v okolju \LaTeX .

Na vse tabele in slike se moramo v besedilu sklicevati, saj kot plovke v oblikovanem besedilu
niso nujno na istem mestu kot v izvornem besedilu.
Pri sklicevanju na slike uporabimo veliko začetnico, npr. ''glej Sliko \ref{pic1}'',
saj gre za ime slike.

\section{Formati slik}

V dokument \LaTeX\ lahko vključimo slike različnih formatov, tako
bitne slike kot vektorske slike. Najbolj primerne so slike v formatu {\tt .pdf}, saj je tudi samo
oblikovano besedilo v tem formatu, in slike v formatu {\tt .jpeg}.
Slika~\ref{pic1} je npr. v formatu {\tt .jpeg}.

\begin{figure}[htb]
    \begin{center}
        \includegraphics[width=0.7\textwidth]{resources/images/projections}
    \end{center}
    \caption{Virtualno obogatena skulptura \cite{vodnjak}.
        Rezultate računalniško generirane animacije z video projektorjem projeciramo na kamnito
        skulpturo, da ustvarimo vtis, kot da bi po skulpturi polzele vodne kapljice
        \cite{video,solina2020skulpture}.}
    \label{pic1}
\end{figure}



\section{Podnapisi k slikam in tabelam}

Vsaki sliki ali tabeli moramo dodati podnapis, ki na kratko pojasnjuje, kaj je na sliki ali tabeli.
Če nekdo le prelista diplomsko delo, naj bi že iz slik in njihovih podnapisov lahko na grobo razbral, kakšno temo naloga obravnava.

Če slike povzamemo iz drugih virov, potem se moramo v podnapisu k taki sliki sklicevati na ta vir!
