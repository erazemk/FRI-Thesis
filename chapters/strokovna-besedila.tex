\chapter{Struktura strokovnih besedil}
\label{stroka}

Strokovna besedila imajo ustaljeno strukturo, da bi lahko hitreje in lažje brali in predvsem
razumeli taka besedila, saj načeloma vemo vnaprej, kje v besedilu se naj bi nahajale določene
informacije.

Najbolj osnovna struktura strokovnega besedila je:
\begin{description}
    \item[naslov besedila,] ki naj bo sicer kratek, a kljub temu dovolj poveden o vsebini besedila,
    \item[imena avtorjev] so običajno navedena po teži prispevka, prvi avtor je tisti, ki je
        besedilo dejansko pisal, zadnji pa tisti, ki je raziskavo vodil,
    \item[kontaktni podatki] -- poleg imena in naslova institucije je potreben vsaj naslov
        elektronske pošte,
    \item[povzetek] je kratko besedilo, ki povsem samostojno povzame vsebino in izpostavi
        predvsem glavne rezultate ali zaključke,
    \item[ključne besede] so tudi namenjene iskanju vsebin med množico člankov,
    \item[uvodno poglavje] uvede bralca v tematiko besedila, razloži kaj je namen besedila,
        predstavi področje o katerem besedilo piše
        (če temu ni namenjeno v celoti posebno poglavje) ter na kratko predstavi strukturo
        celotnega besedila,
    \item[poglavja] tvorijo zaokrožene celote, ki se po potrebi še nadalje členijo na podpoglavja,
        namenjena so recimo opisu orodij, ki smo jih uporabili pri delu, teoretičnim rezultatom ali
        predstavitvi rezultatov, ki smo jih dosegli,
    \item[zaključek] še enkrat izpostavi glavne rezultate ali ugotovitve, jih primerja z
        dosedanjimi in morebiti poda tudi ideje za nadaljno delo,
    \item[literatura] je seznam vseh virov, na katere smo se pri svojem delu opirali, oziroma
        smo se na njih sklicevali v svojem besedilu.
\end{description}

Naslove poglavij in podpoglavij izbiramo tako, da lahko bralec že pri prelistavanju diplome in
branju naslovov v grobem ugotovi, kaj je vsebina diplomskega dela.

Strokovna besedila običajno pišemo v prvi osebi množine, v nevtralnem in umirjenem tonu.
Uporaba sopomenk ni zaželena, saj želimo zaradi lažjega razumevanja za iste pojme vseskozi
uporabljati iste besede.
Najpomembnejše ugotovitve je smiselno večkrat zapisati, na primer v povzetku, uvodu, glavnem
delu in zaključku.
Vse trditve naj bi temeljile bodisi na lastnih ugotovitvah (izpeljavah, preizkusih, testiranjih)
ali pa z navajanjem ustreznih virov.

Največ se lahko naučimo s skrbnim branjem dobrih zgledov takih besedil.
